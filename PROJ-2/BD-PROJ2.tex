\documentclass[11pt,a4paper]{article}
 
\usepackage[T1]{fontenc}
\usepackage[portuguese]{babel}
\usepackage{authblk}
\usepackage{listings}
\usepackage{enumitem}
\usepackage{graphicx}
\usepackage[hmargin=2cm,vmargin=3.5cm,bmargin=3cm]{geometry}
\usepackage[latin9]{inputenc}
\usepackage{eurosym}\def\texteuro{\euro}


\newcommand{\select}{\mbox{\Large$\sigma$}}

\pagenumbering{arabic}

\title{\textbf{Projecto de Bases de Dados, Parte 2}}
 
\author{Bruno Cardoso (72619), L�dia Freitas (78559) e Rodrigo Bernardo (78942)
} 

\affil{Instituto Superior T�cnico}

\begin{document}

\date{11 de Dezembro de 2015}

\maketitle

\centerline{\includegraphics[width=0.4\textwidth]{ist-simbolo.jpg}}

\begin{description}[noitemsep]
	\item \centering{Grupo 17}
	\item Turno: Quinta-Feira, 08h00, LAB 14
	\item 25 horas de trabalho por aluno.
\end{description}


\newpage

\tableofcontents
\newpage

\section{Consultas SQL}
\newpage
\section{Restri��es de Integridade}
\newpage
\section{Formas Normais}

\paragraph{}
(a) A rela��o utilizador, tem apenas as depend�ncias funcionais (DFs) da forma X -> A,
com A pertencente aos atributos desta relacao e X � {userid, email}. Como em todas
estas DFs se tem que o determinante � chave, a rela��o utilizador encontra-se na
Boyce-Codd Normal Form (BCNF).
(b)
(1) Para al�m das DFs anteriores, a rela��o tem agora uma nova DF na qual o determinante
n�o � chave, mas o dependente �. Assim, a rela��o utilizador encontra-se na terceira forma
normal.
(2) R1(\_nome, email, \_password, \_questao1, \_resposta1, \_questao2, \_resposta2)
     R2(\_userid, nome, password, questao1, resposta1, questao2, resposta2, pais, categoria)


\newpage
\section{�ndices}
\newpage
\section{Transac��es}
\newpage
\section{Data Warehouse}


\end{document}